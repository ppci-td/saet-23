No mês que vem começa a grande Copa do Mundo de Dados Estruturados de 2022, que
este ano acontece na grande cidade de Touledow. Para motivar as pessoas a
acompanherem o torneio\footnote{o que os jovens de hoje chamam de ``hypar'' o
evento}, diversas ações de publicidade e \textit{merchandising}
estão sendo feitas durante o ano todo.

Em uma dessas ações, a organização lançou o Álbum de Figurinhas Oficial da
Copa$^\circledR$. O álbum é composto por $N$ figurinhas distintas, numeradas de
1 a $N$. Para completar o álbum, é necessário obter pelo menos uma cópia de cada
uma dessas $N$ figurinhas.

Ana tem uma coleção de figurinhas que pode não conter todas as figurinhas de 1
a $N$, e pode conter algumas figurinhas repetidas. Beto também tem uma coleção
de figurinhas com as mesmas condições.

Para poderem completar seus álbuns, Ana e Beto irão trocar figurinhas.
Em uma troca, Ana dá uma figurinha para Beto e, ao mesmo tempo,
Beto dá uma figurinha para Ana. Note que apenas \textit{trocas} são
permitidas -- uma pessoa não irá dar uma figurinha para outra sem receber uma
outra figurinha em troca.

Determine o menor número de trocas necessário para que ambos completem seus
álbuns, ou indique que isso não é possível.

\subsection*{Entrada}

A primeira linha contém o inteiro $N$ ($1 \leq N \leq 1000$).

A próxima linha
contém $N_A$ ($1 \leq N_A \leq 1000$), o número de figurinhas de Ana.

A próxima
linha contém $N_A$ inteiros de 1 a $N$, indicando quais figurinhas Ana tem.

A próxima linha
contém $N_B$ ($1 \leq N_B \leq 1000$), o número de figurinhas de Beto.

A próxima
linha contém $N_B$ inteiros de 1 a $N$, indicando quais figurinhas Beto tem.

\subsection*{Saída}

Imprima uma única linha contendo o número mínimo de trocas necessárias, ou
\verb|impossivel| se não for possível que ambos completem seus álbuns trocando
figurinhas.

\newpage
\begin{table}[!h]
\centering
\begin{tabular}{|l|l|}
\hline
\begin{minipage}[t]{3in}
\textbf{Exemplo de entrada}
\begin{verbatim}
5
5
1 1 2 4 5
5
2 3 3 4 5
\end{verbatim}
\vspace{1mm}
\end{minipage}
&
\begin{minipage}[t]{3in}
\textbf{Exemplo de saída}
\begin{verbatim}
1
\end{verbatim}
\vspace{1mm}
\end{minipage} \\
\hline
\end{tabular}
\end{table}

\begin{table}[!h]
\centering
\begin{tabular}{|l|l|}
\hline
\begin{minipage}[t]{3in}
\textbf{Exemplo de entrada}
\begin{verbatim}
5
6
1 1 2 4 5 5
5
2 3 3 3 4
\end{verbatim}
\vspace{1mm}
\end{minipage}
&
\begin{minipage}[t]{3in}
\textbf{Exemplo de saída}
\begin{verbatim}
2
\end{verbatim}
\vspace{1mm}
\end{minipage} \\
\hline
\end{tabular}
\end{table}

\begin{table}[!h]
\centering
\begin{tabular}{|l|l|}
\hline
\begin{minipage}[t]{3in}
\textbf{Exemplo de entrada}
\begin{verbatim}
5
6
1 1 2 4 5 5
4
2 3 3 4
\end{verbatim}
\vspace{1mm}
\end{minipage}
&
\begin{minipage}[t]{3in}
\textbf{Exemplo de saída}
\begin{verbatim}
impossivel
\end{verbatim}
\vspace{1mm}
\end{minipage} \\
\hline
\end{tabular}
\end{table}

\begin{table}[!h]
\centering
\begin{tabular}{|l|l|}
\hline
\begin{minipage}[t]{3in}
\textbf{Exemplo de entrada}
\begin{verbatim}
10
13
9 1 7 1 3 3 5 3 5 9 7 7 9
12
6 2 4 2 10 4 8 4 6 8 6 10
\end{verbatim}
\vspace{1mm}
\end{minipage}
&
\begin{minipage}[t]{3in}
\textbf{Exemplo de saída}
\begin{verbatim}
5
\end{verbatim}
\vspace{1mm}
\end{minipage} \\
\hline
\end{tabular}
\end{table}
