Este ano temos eleições gerais para escolher o novo chefe de governo da Nlogonia.

Cada um dos $N$ eleitores do país tem três opções de como
votar no dia da eleição: votar em um dos $M$
candidatos que estão concorrendo ao cargo; votar em branco; ou anular seu
voto.

Um \textit{voto válido} é um voto dado em algum dos $M$ candidatos (isto é, um voto
que não é nem em branco nem nulo). Se algum candidato receber mais que $50\%$
de todos os votos válidos, ele vence a eleição. Caso contrário,
os dois candidatos que mais receberem votos válidos irão disputar uma nova
eleição, em segundo turno.

Chegou a hora de totalizar os votos! Dados os votos registrados nas urnas, algum
candidato venceu a eleição? Ou haverá segundo turno?

\subsection*{Entrada}

A primeira linha contém dois inteiros $N$ e $M$
($1 \leq N, M \leq 10^5$), o número de eleitores e de candidatos,
respectivamente. Os candidatos são numerados de $1$ a $M$.
A próxima linha contém $N$ votos. Cada voto pode ser um número de um cadidato, a
letra \verb|B| (voto em branco) ou a letra \verb|N| (voto nulo).

É garantido que a entrada contém no mínimo 1 voto válido, e que, se houver
segundo turno, não haverá empate nem na primeira nem na segunda colocação.

\subsection*{Saída}

Se não houver segundo turno, imprima uma linha contendo
\verb|vencedor |$C$, sendo $C$ o número do candidato vencedor. Caso contrário,
    imprima uma linha contendo \verb|segundo turno entre |$A$\verb| e |$B$, onde
    $A$ e $B$ são os números do candidato mais votado e o segundo mais votado,
    respectivamente.


\begin{table}[!h]
\centering
\begin{tabular}{|l|l|}
\hline
\begin{minipage}[t]{3in}
\textbf{Exemplo de entrada}
\begin{verbatim}
7 100
92 92 B 7 92 7 N
\end{verbatim}
\vspace{1mm}
\end{minipage}
&
\begin{minipage}[t]{3in}
\textbf{Exemplo de saída}
\begin{verbatim}
vencedor 92
\end{verbatim}
\vspace{1mm}
\end{minipage} \\
\hline
\end{tabular}
\end{table}

\begin{table}[!h]
\centering
\begin{tabular}{|l|l|}
\hline
\begin{minipage}[t]{3in}
\textbf{Exemplo de entrada}
\begin{verbatim}
11 100
B 95 N 95 95 99 99 B 95 98 99
\end{verbatim}
\vspace{1mm}
\end{minipage}
&
\begin{minipage}[t]{3in}
\textbf{Exemplo de saída}
\begin{verbatim}
segundo turno entre 95 e 99
\end{verbatim}
\vspace{1mm}
\end{minipage} \\
\hline
\end{tabular}
\end{table}

\begin{table}[!h]
\centering
\begin{tabular}{|l|l|}
\hline
\begin{minipage}[t]{3in}
\textbf{Exemplo de entrada}
\begin{verbatim}
10 10
9 B 5 N 5 9 B 5 9 5
\end{verbatim}
\vspace{1mm}
\end{minipage}
&
\begin{minipage}[t]{3in}
\textbf{Exemplo de saída}
\begin{verbatim}
vencedor 5
\end{verbatim}
\vspace{1mm}
\end{minipage} \\
\hline
\end{tabular}
\end{table}
