Muito antes de Alan Turing (1912--1954)\footnote{Obrigado Alan Turing, ateu,
homossexual e pai da Computação} decifrar o esquema de criptografia usado
pelos alemães na segunda guerra mundial, Júlio César já havia criado seu próprio
algoritmo de criptografia para uso militar durante a República Romana.

Dado um inteiro $K$, a Cifra de César criptografa uma mensagem substituindo cada
letra da mensagem pela letra que aparece $K$ posições a sua frente na ordem
alfabética (ciclicamente). Por exemplo, se $K=2$, todo \verb|A| é substituído por
\verb|C|;
todo \verb|B| é substituído por \verb|D|; ...; todo \verb|X| é substituído por
\verb|Z|; todo \verb|Y| é
substituído por \verb|A|; e todo \verb|Z| é substituído por \verb|B|. Desta forma, se $K=2$, a
mensagem \verb|ATAQUE|, por exemplo, é criptografada em \verb|CVCSWG|.

Você é um soldado espião inimigo da República, e sua tarefa é descriptografar
mensagens criptografadas pela Cifra de César.

\subsection*{Entrada}

A primeira linha contém o valor de $K$ utilizado na criptografia $(0 \leq K <
        26$). A segunda linha contém a mensagem criptografada. A mensagem contém
de 1 até 1000 letras maiúsculas.

\subsection*{Saída}

Imprima uma única linha contendo a mensagem descriptografada.

\begin{table}[!h]
\centering
\begin{tabular}{|l|l|}
\hline
\begin{minipage}[t]{3in}
\textbf{Exemplo de entrada}
\begin{verbatim}
2
CVCSWG
\end{verbatim}
\vspace{1mm}
\end{minipage}
&
\begin{minipage}[t]{3in}
\textbf{Exemplo de saída}
\begin{verbatim}
ATAQUE
\end{verbatim}
\vspace{1mm}
\end{minipage} \\
\hline
\end{tabular}
\end{table}

\begin{table}[!h]
\centering
\begin{tabular}{|l|l|}
\hline
\begin{minipage}[t]{3in}
\textbf{Exemplo de entrada}
\begin{verbatim}
13
OBENOVYY
\end{verbatim}
\vspace{1mm}
\end{minipage}
&
\begin{minipage}[t]{3in}
\textbf{Exemplo de saída}
\begin{verbatim}
BORABILL
\end{verbatim}
\vspace{1mm}
\end{minipage} \\
\hline
\end{tabular}
\end{table}

\begin{table}[!h]
\centering
\begin{tabular}{|l|l|}
\hline
\begin{minipage}[t]{3in}
\textbf{Exemplo de entrada}
\begin{verbatim}
20
BYFFIQILFX
\end{verbatim}
\vspace{1mm}
\end{minipage}
&
\begin{minipage}[t]{3in}
\textbf{Exemplo de saída}
\begin{verbatim}
HELLOWORLD
\end{verbatim}
\vspace{1mm}
\end{minipage} \\
\hline
\end{tabular}
\end{table}
