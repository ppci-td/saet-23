\vspace{-0.5cm}
Sempre que o prof. Racirdinho fala sobre números pares e números ímpares durante
as aulas de Teoria da Poncutação, uma dúvida geral surge: ``o número 0 (zero) é
par?''.

Embora o professor insista que zero seja sim um número par, Mariazinha ainda está
desconfiada desta informação, e prefere não tomar isso como verdade. Para não
gerar conflitos, Mariazinha decidiu criar sua própria definição de números pares.

Um número é \textit{mar} (abreviação para ``Mariazinha-par'') se é positivo, é
múltiplo de dois, e não contém nenhum 0 (zero) em sua representação decimal.
Como exemplo, os números 2, 42, e 796 são mares, enquanto 0, 20, 402 e 7096 não são
mares.

Os primeiros números mares são, nesta ordem, 2, 4, 6, 8, 12, 14, 16, 18, 22, 24, etc.
Sua tarefa é, para várias consultas, determinar, dado um inteiro $N$, qual é o $N-$ésimo número mar.

\subsection*{Entrada}

A primeira linha contém um inteiro $Q$ ($1 \leq Q \leq 100$), o número de
consultas. As próximas $Q$ linhas descrevem uma consulta cada.
Cada linha contém um inteiro $N$ ($1 \leq N \leq 10^{15}$).

\subsection*{Saída}

Para cada consulta, imprima uma linha contendo o $N-$ésimo número mar.

\begin{table}[!h]
\centering
\begin{tabular}{|l|l|}
\hline
\begin{minipage}[t]{3in}
\textbf{Exemplo de entrada}
\begin{verbatim}
6
1
2
10
100
38
1000
\end{verbatim}
\vspace{1mm}
\end{minipage}
&
\begin{minipage}[t]{3in}
\textbf{Exemplo de saída}
\begin{verbatim}
2
4
24
268
94
2968
\end{verbatim}
\vspace{1mm}
\end{minipage} \\
\hline
\end{tabular}
\end{table}
