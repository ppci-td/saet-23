Enquanto preparava suas anotações para levar para a Maratona de Programação,
Douglas percebeu um fato interessante: algumas estruturas de dados fazem tudo o
que outras estruturas fazem e mais um pouco, e logo poderiam substitui-las sem
prejuizo. É o caso, por exemplo, da BIT (\textit{Binary Indexed Tree}) e da Seg
(\textit{Segment Tree}). A Seg tem todas as funcionalidades que a BIT tem (além
        das funcionalidades que só a Seg possui). Por isso, conhecer a
BIT não é \textit{realmente} necessário quando se já conhece a Seg\footnote{embora é recomendado, pois o
código da BIT é mais compacto que o da Seg, tornando-a mais prática.}, pois todo problema que pode
ser resolvido com a BIT pode ser resolvido com a Seg em seu lugar. Neste caso,
    dizemos que a Seg \textit{pode substituir} a BIT.

Após estudar todas as $N$ estruturas de dados que existem, Douglas descobriu quais
estruturas podem substituir quais outras. Agora Douglas está curioso: qual é a estrutura
de dados que pode substituir (direta ou indiretamente) a maior quantidade de outras estruturas?

\subsection*{Entrada}

A primeira linha contém dois inteiros $N$ e $M$
($1 \leq N \leq 2000, 0 \leq M \leq min(\frac{N^2 - N}{2}, 2000$)), o número de estruturas e de substituições,
respectivamente. As estruturas são numeradas de $1$ a $N$.
As próximas $M$ linhas contém dois inteiros $A$ e $B$ cada
$(1 \leq A, B \leq N, A \neq B)$, indicando que a estrutura $A$ pode substituir a
estrutura $B$.

É garantido que a relação de substituições não contém ciclos.

\subsection*{Saída}

Imprima uma linha com dois inteiros $E$ e $S$, onde $E$ é a estrutura de dados
que pode substituir (direta ou indiretamente) a maior quantidade de outras estruturas, enquanto $S$ é a quantidade
de outras estruturas que $E$ pode substituir (direta ou indiretamente).

Se houver mais de uma estrutura que pode substituir a maior quantidade de estruturas,
imprima a de menor número identificador.

\vspace{-0.3cm}
\begin{table}[!h]
\centering
\begin{tabular}{|l|l|}
\hline
\begin{minipage}[t]{3in}
\textbf{Exemplo de entrada}
\begin{verbatim}
8 9
1 2
2 5
2 6
5 6
3 2
7 6
3 7
4 3
8 7
\end{verbatim}
\vspace{1mm}
\end{minipage}
&
\begin{minipage}[t]{3in}
\textbf{Exemplo de saída}
\begin{verbatim}
4 5
\end{verbatim}
\vspace{1mm}
\end{minipage} \\
\hline
\end{tabular}
\end{table}
