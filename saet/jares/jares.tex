\vspace{-0.5cm}
Sempre que o prof. Racirdinho fala sobre números pares e números ímpares durante
as aulas de Teoria da Poncutação, uma dúvida geral surge: ``o número 0 (zero) é
par?''.

Embora o professor insista que zero seja sim um número par, Joãozinho ainda está
desconfiado desta informação, e prefere não tomar isso como verdade. Para não
gerar conflitos, Joãozinho decidiu criar sua própria definição de números pares.

Um número é \textit{jar} (abreviação para ``Joãozinho-par'') se é positivo, é
múltiplo de dois, e não termina com zero em sua representação decimal.
Desta forma, os primeiros números jares são 2, 4, 6, 8, 12, 14, 16, 18, 22, ....
Dado um número inteiro, determine se ele é um número jar.

\subsection*{Entrada}

A única linha da entrada contém um inteiro $N$ ($-10^9 \leq N \leq 10^9$).

\subsection*{Saída}

Imprima uma única linha contendo \verb|sim| se $N$ é um número jar, ou
\verb|nao| caso contrário.

\begin{table}[!h]
\centering
\begin{tabular}{|l|l|}
\hline
\begin{minipage}[t]{3in}
\textbf{Exemplo de entrada}
\begin{verbatim}
0
\end{verbatim}
\vspace{1mm}
\end{minipage}
&
\begin{minipage}[t]{3in}
\textbf{Exemplo de saída}
\begin{verbatim}
nao
\end{verbatim}
\vspace{1mm}
\end{minipage} \\
\hline
\end{tabular}
\end{table}

\begin{table}[!h]
\centering
\begin{tabular}{|l|l|}
\hline
\begin{minipage}[t]{3in}
\textbf{Exemplo de entrada}
\begin{verbatim}
42
\end{verbatim}
\vspace{1mm}
\end{minipage}
&
\begin{minipage}[t]{3in}
\textbf{Exemplo de saída}
\begin{verbatim}
sim
\end{verbatim}
\vspace{1mm}
\end{minipage} \\
\hline
\end{tabular}
\end{table}

\begin{table}[!h]
\centering
\begin{tabular}{|l|l|}
\hline
\begin{minipage}[t]{3in}
\textbf{Exemplo de entrada}
\begin{verbatim}
100
\end{verbatim}
\vspace{1mm}
\end{minipage}
&
\begin{minipage}[t]{3in}
\textbf{Exemplo de saída}
\begin{verbatim}
nao
\end{verbatim}
\vspace{1mm}
\end{minipage} \\
\hline
\end{tabular}
\end{table}

\begin{table}[!h]
\centering
\begin{tabular}{|l|l|}
\hline
\begin{minipage}[t]{3in}
\textbf{Exemplo de entrada}
\begin{verbatim}
102
\end{verbatim}
\vspace{1mm}
\end{minipage}
&
\begin{minipage}[t]{3in}
\textbf{Exemplo de saída}
\begin{verbatim}
sim
\end{verbatim}
\vspace{1mm}
\end{minipage} \\
\hline
\end{tabular}
\end{table}

\begin{table}[!h]
\centering
\begin{tabular}{|l|l|}
\hline
\begin{minipage}[t]{3in}
\textbf{Exemplo de entrada}
\begin{verbatim}
642487
\end{verbatim}
\vspace{1mm}
\end{minipage}
&
\begin{minipage}[t]{3in}
\textbf{Exemplo de saída}
\begin{verbatim}
nao
\end{verbatim}
\vspace{1mm}
\end{minipage} \\
\hline
\end{tabular}
\end{table}

\begin{table}[!h]
\centering
\begin{tabular}{|l|l|}
\hline
\begin{minipage}[t]{3in}
\textbf{Exemplo de entrada}
\begin{verbatim}
-2
\end{verbatim}
\vspace{1mm}
\end{minipage}
&
\begin{minipage}[t]{3in}
\textbf{Exemplo de saída}
\begin{verbatim}
nao
\end{verbatim}
\vspace{1mm}
\end{minipage} \\
\hline
\end{tabular}
\end{table}
