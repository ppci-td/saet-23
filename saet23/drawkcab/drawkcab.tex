Uma das mais divertidas tarefas de um maratonista de programação é a escolha do
nome da sua equipe. Depois de muito debate, sua equipe decidiu escolher o nome
da seguinte maneira:

\begin{itemize}
    \item O nome deverá ser uma substring não vazia de uma dada string $s$;
    \item O nome deverá ser palíndrome (isto é, deve ser igual quando lido da
            esquerda para a direita e da direita para a esquerda);
    \item Cada letra do alfabeto $a$, $b$, $c$, ..., $z$ tem um \textit{valor}
    $V_a$, $V_b$, $V_c$, ..., $V_z$. O \textit{valor total} do nome escolhido é a soma dos valores de
    suas letras. O nome escolhido deverá ter o maior valor total possível.
\end{itemize}

Como exemplo, considere a string $s = $ \verb|xabaydcbbcdyz| e os valores
$V_a = 20$, $V_b = -10$, $V_c = 15$, $V_d = 11$, $V_x = 20$, $V_y = -20$ e $V_z
= 20$. Alguns nomes que poderiam ser escolhidos são:

\begin{itemize}
    \item \verb|aba|, com valor total $V_a + V_b + V_a = 20-10+20 = 30$;
    \item \verb|dcbbcd|, com valor total $11 + 15 - 10 - 10 + 15 + 11 = 32$;
    \item \verb|ydcbbcdy|, com valor total $-20 + 11 + 15 - 10 - 10 + 15 + 11
    -20 = -8$;
    \item Outras substrings palíndromes de $s$.
\end{itemize}

Neste exemplo, a substring palíndrome \verb|dcbbcd| tem o maior valor total
possível (32), e portanto este será o nome da equipe.

Dada a string $s$ e os valores de cada letra, ajude a escolher o nome da sua
equipe!

\subsection*{Entrada}

A primeira linha contém 26 valores inteiros $V_a, V_b, V_c, ..., V_z$ (entre
        $-1000$ e $1000$ cada) indicando o
valor de cada letra do alfabeto.

A segunda linha contém a string $s$ ($1 \leq |s| \leq 5 \times 10^5$), contendo
apenas letras minúsculas.

\subsection*{Saída}

Imprima uma única linha contendo o valor total do nome escolhido pela equipe.

\newpage
\begin{table}[!h]
\centering
\hspace{-2cm}
\begin{tabular}{|l|l|}
\hline
\begin{minipage}[t]{5.5in}
\textbf{Exemplo de entrada}
\begin{verbatim}
20 -10 15 11 0 0 0 0 0 0 0 0 0 0 0 0 0 0 0 0 0 0 20 0 -20 20
xabaydcbbcdyz
\end{verbatim}
\vspace{1mm}
\end{minipage}
&
\begin{minipage}[t]{1.5in}
\textbf{Exemplo de saída}
\begin{verbatim}
32
\end{verbatim}
\vspace{1mm}
\end{minipage} \\
\hline
\end{tabular}
\end{table}

\begin{table}[!h]
\centering
\hspace{-2cm}
\begin{tabular}{|l|l|}
\hline
\begin{minipage}[t]{5.5in}
\textbf{Exemplo de entrada}
\begin{verbatim}
1 -1 0 0 0 0 0 0 0 0 0 0 0 0 0 0 0 0 0 0 0 0 0 0 0 0
bbbaaaaaabbbbbbaaa
\end{verbatim}
\vspace{1mm}
\end{minipage}
&
\begin{minipage}[t]{1.5in}
\textbf{Exemplo de saída}
\begin{verbatim}
6
\end{verbatim}
\vspace{1mm}
\end{minipage} \\
\hline
\end{tabular}
\end{table}

\begin{table}[!h]
\centering
\hspace{-2cm}
\begin{tabular}{|l|l|}
\hline
\begin{minipage}[t]{5.5in}
\textbf{Exemplo de entrada}
\begin{verbatim}
7 1 1 8 1 -5 1 1 3 1 -10 1 71 -42 1 1 1 1 1 1 1 1 1 1 1 1
lazafakeekafdrawkcabackwardlaza
\end{verbatim}
\vspace{1mm}
\end{minipage}
&
\begin{minipage}[t]{1.5in}
\textbf{Exemplo de saída}
\begin{verbatim}
31
\end{verbatim}
\vspace{1mm}
\end{minipage} \\
\hline
\end{tabular}
\end{table}

\begin{table}[!h]
\centering
\hspace{-2cm}
\begin{tabular}{|l|l|}
\hline
\begin{minipage}[t]{5.5in}
\textbf{Exemplo de entrada}
\begin{verbatim}
-1 0 0 0 0 0 0 0 0 0 0 0 0 0 0 0 0 0 0 0 0 0 0 0 0 0
aaaaaaaaaaaaaaaaaaaaaaaaaaaaaaaaaaaaaaaaaaaaaaaaaaaa
\end{verbatim}
\vspace{1mm}
\end{minipage}
&
\begin{minipage}[t]{1.5in}
\textbf{Exemplo de saída}
\begin{verbatim}
-1
\end{verbatim}
\vspace{1mm}
\end{minipage} \\
\hline
\end{tabular}
\end{table}
