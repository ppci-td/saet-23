\documentclass{article}
\usepackage{amsmath}
\usepackage{color,pxfonts,fix-cm}
\usepackage{latexsym}
\usepackage[mathletters]{ucs}
\usepackage[T1]{fontenc}
\usepackage[utf8x]{inputenc}
\usepackage{pict2e}
\usepackage{wasysym}
\usepackage[english]{babel}
\usepackage{tikz}
\pagestyle{empty}
\usepackage[margin=0in,paperwidth=596pt,paperheight=842pt]{geometry}
\begin{document}
\definecolor{color_283006}{rgb}{1,1,1}
\definecolor{color_29791}{rgb}{0,0,0}
\begin{tikzpicture}[overlay]
\path(0pt,0pt);
\filldraw[color_283006][nonzero rule]
(-15pt, 10pt) -- (580.5pt, 10pt)
 -- (580.5pt, 10pt)
 -- (580.5pt, -832.25pt)
 -- (580.5pt, -832.25pt)
 -- (-15pt, -832.25pt) -- cycle
;
\end{tikzpicture}
\begin{picture}(-5,0)(2.5,0)
\put(57,-73.25586){\fontsize{12}{1}\usefont{T1}{cmr}{m}{n}\selectfont\color{color_29791}Um professor de matemática que é fascinado por competições resolveu fazer uma}
\put(57,-93.9541){\fontsize{12}{1}\usefont{T1}{cmr}{m}{n}\selectfont\color{color_29791}dinâmica com a sua turma. Ele percebeu que em sua casa havia uma grande}
\put(57,-114.6523){\fontsize{12}{1}\usefont{T1}{cmr}{m}{n}\selectfont\color{color_29791}quantidade de copos e optou por utilizá-los em sua próxima aula. Ele explicou da}
\put(57,-135.3506){\fontsize{12}{1}\usefont{T1}{cmr}{m}{n}\selectfont\color{color_29791}seguinte maneira para os seus alunos: dado N copos (1 <= N <= 10000), eles}
\put(57,-156.0488){\fontsize{12}{1}\usefont{T1}{cmr}{m}{n}\selectfont\color{color_29791}devem formar um triângulo de base B tal que a quantidade de copos a partir da}
\put(57,-176.7471){\fontsize{12}{1}\usefont{T1}{cmr}{m}{n}\selectfont\color{color_29791}base seja igual a B.}
\put(62.25,-244.8418){\fontsize{12}{1}\usefont{T1}{cmr}{m}{n}\selectfont\color{color_29791}Entrada}
\put(287.25,-244.8418){\fontsize{12}{1}\usefont{T1}{cmr}{m}{n}\selectfont\color{color_29791}Saída}
\put(62.25,-269.8906){\fontsize{12}{1}\usefont{T1}{cmr}{m}{n}\selectfont\color{color_29791}10}
\put(287.25,-269.8906){\fontsize{12}{1}\usefont{T1}{cmr}{m}{n}\selectfont\color{color_29791}4}
\put(62.25,-314.1406){\fontsize{12}{1}\usefont{T1}{cmr}{m}{n}\selectfont\color{color_29791}8}
\put(287.25,-314.1406){\fontsize{12}{1}\usefont{T1}{cmr}{m}{n}\selectfont\color{color_29791}3}
\put(62.25,-359.1406){\fontsize{12}{1}\usefont{T1}{cmr}{m}{n}\selectfont\color{color_29791}2}
\put(287.25,-359.1406){\fontsize{12}{1}\usefont{T1}{cmr}{m}{n}\selectfont\color{color_29791}1}
\put(62.25,-404.8906){\fontsize{12}{1}\usefont{T1}{cmr}{m}{n}\selectfont\color{color_29791}19}
\put(287.25,-404.8906){\fontsize{12}{1}\usefont{T1}{cmr}{m}{n}\selectfont\color{color_29791}5}
\put(62.25,-452.8906){\fontsize{12}{1}\usefont{T1}{cmr}{m}{n}\selectfont\color{color_29791}1000}
\put(287.25,-452.8906){\fontsize{12}{1}\usefont{T1}{cmr}{m}{n}\selectfont\color{color_29791}44}
\end{picture}
\begin{tikzpicture}[overlay]
\path(0pt,0pt);
\draw[color_29791,line width=1pt]
(57.5pt, -227pt) -- (57.5pt, -486pt)
;
\draw[color_29791,line width=1pt]
(282.5pt, -227pt) -- (282.5pt, -486pt)
;
\draw[color_29791,line width=1pt]
(507.5pt, -227pt) -- (507.5pt, -486pt)
;
\draw[color_29791,line width=1pt]
(57pt, -227.5pt) -- (507pt, -227.5pt)
;
\draw[color_29791,line width=1pt]
(57pt, -252.5pt) -- (507pt, -252.5pt)
;
\draw[color_29791,line width=1pt]
(57pt, -296.5pt) -- (507pt, -296.5pt)
;
\draw[color_29791,line width=1pt]
(57pt, -341.5pt) -- (507pt, -341.5pt)
;
\draw[color_29791,line width=1pt]
(57pt, -387.5pt) -- (507pt, -387.5pt)
;
\draw[color_29791,line width=1pt]
(57pt, -435.5pt) -- (507pt, -435.5pt)
;
\draw[color_29791,line width=1pt]
(57pt, -485.5pt) -- (507pt, -485.5pt)
;
\end{tikzpicture}
\end{document}