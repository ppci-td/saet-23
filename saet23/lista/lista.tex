Pedro tem uma clínica que funciona 7 dias por semana, porém ele não gosta
de trabalhar muito aos finais de semana, e por isso pediu para sua secretária
distribuir os seus $N$ pacientes de acordo com a primeira letra de seus nomes em
certos dias da semana.

Ele atende as pessoas que têm os nomes que começam
com as letras A,B,C e D no domingo; já na segunda-feira ele atende as pessoas que têm
os nomes que começar com as letras E,F,G,e H; e assim por diante até
a sexta-feira, quando só atende as pessoas com as letras U, V e W , e no sábado X,
Y e Z .

Entretanto, ele pediu para sua secretária que não organizasse os pacientes,
    dentro de um mesmo dia, em ordem
alfabética, pois assim seria injusto com os pacientes que marcaram suas consultas
antes. Dentro de um mesmo dia, a lista de pacientes deve seguir a ordem em que
marcaram as consultas.

Assim, por exemplo, se Caio, Felipe, Diego,
Ana, Bianca, Alice e Tiago marcaram consultas (nessa ordem),
seriam atendidos no domingo, nessa ordem: Caio, Diego, Ana, Bianca e Alice;
Já Felipe seria atendido na segunda-feira, portanto após todos esses; Tiago
seria atendido por último.

\subsection*{Entrada}

A primeira linha contém $N$ ($1 \leq N \leq 1000$), o número de pacientes. As
próximas $N$ linhas contém um nome cada, na ordem em que marcaram as consultas.
Os nomes serão strings de até 20 letras minsculas e/ou maiusculas cada.

\subsection*{Saída}

Imprima um nome de paciente por linha, na ordem em que serão atendidos.

%----- Exemplo 1 -----%
\begin{table}[!h]
\centering
\begin{tabular}{|l|l|}
\hline
\begin{minipage}[t]{3in}
\textbf{Exemplo de entrada}
\begin{verbatim}
5
bruna
Gabi
ana
Caio
bianca
\end{verbatim}
\vspace{1mm}
\end{minipage}
&
\begin{minipage}[t]{3in}
\textbf{Exemplo de saída}
\begin{verbatim}
bruna
ana
Caio
bianca
Gabi
\end{verbatim}
\vspace{1mm}
\end{minipage} \\
\hline
\end{tabular}
\end{table}
