Seu melhor amigo, Gabriel, te pediu ajuda para criar a melhor estratégia para a raid que ele irá participar em LIT (Lost In Time), o jogo favorito dele. Ele te explicou como a raid funciona da seguinte forma:

\begin{itemize}
    \item Para causar dano nos mobs você deve usar magias.
    \item Para cada magia usada você causa exatamente o dano dela no monstro e o monstro é eliminado quando ele tiver 0 HP.
    \item Caso você cause mais dano do que o necessário para eliminar o monstro você recebe uma nova magia com o valor da diferença no dano.
\end{itemize}

Para se preparar para a raid ele deve comprar as magias usando as LIT'coins.
\begin{itemize}
    \item As magias disponivéis estão ordenadas linearmente na loja.
    \item Cada magia custa exatamente seu valor em dano de LIT'coins.
    \item É possível comprar multiplas magias somente em um único intervalo linear $[i; j]$ $i \leq j$
    \item Para comprar multiplas magias você deve comprar todas as magias dentro do intervalo $[i; j]$ ($a_i, a_{i+1}, a_{i+2}, ..., a_j$).
\end{itemize}

O pedido de Gabriel é simples, ajude ele determinar qual a menor quantidade possível de LIT'coins para vencer a raid.

\subsection*{Entrada}
A primeira linha contém dois inteiro $N$ e $M$ ($1 \leq M \leq N \leq 10^5$).

A segunda linha contém $N$ inteiros ($1 \leq a_i \leq 100$) que representam as magias disponíveis para a raid.

A terceira linha contém $M$ inteiros ($1 \leq a_i \leq 100$) que representam o HP dos monstros da raid.

\subsection*{Saída}
Imprima a quantidade mínima de LIT'coins para vencer a raid. Caso não seja possível completá-la imprima $-1$.

%----- Exemplo 1 -----%
\begin{table}[!h]
    \centering
    \begin{tabular}{|l|l|}
    \hline
    \begin{minipage}[t]{3in}
    \textbf{Exemplo de entrada}
    \begin{verbatim}
    4 3
    2 3 4 6
    3 1 2 
    \end{verbatim}
    \vspace{1mm}
    \end{minipage}
    &
    \begin{minipage}[t]{3in}
    \textbf{Exemplo de saída}
    \begin{verbatim}
    6
    \end{verbatim}
    \vspace{1mm}
    \end{minipage} \\
    \hline
    \end{tabular}
    \end{table}
    
    %----- Exemplo 2 -----%
    \begin{table}[!h]
    \centering
    \begin{tabular}{|l|l|}
    \hline
    \begin{minipage}[t]{3in}
    \textbf{Exemplo de entrada}
    \begin{verbatim}
    4 2
    3 1 2 6
    5 1
    \end{verbatim}
    \vspace{1mm}
    \end{minipage}
    &
    \begin{minipage}[t]{3in}
    \textbf{Exemplo de saída}
    \begin{verbatim}
    6
    \end{verbatim}
    \vspace{1mm}
    \end{minipage} \\
    \hline
    \end{tabular}
    \end{table}
    
    %----- Exemplo 3 -----%
    \begin{table}[!h]
    \centering
    \begin{tabular}{|l|l|}
    \hline
    \begin{minipage}[t]{3in}
    \textbf{Exemplo de entrada}
    \begin{verbatim}
    4 4
    2 3 3 4
    5 2 1 8
    \end{verbatim}
    \vspace{1mm}
    \end{minipage}
    &
    \begin{minipage}[t]{3in}
    \textbf{Exemplo de saída}
    \begin{verbatim}
    -1
    \end{verbatim}
    \vspace{1mm}
    \end{minipage} \\
    \hline
    \end{tabular}
    \end{table}