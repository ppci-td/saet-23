Sila trabalha em um pequeno ateliê de costura próprio onde é confeccionado de tudo, desde camisetas e uniformes até o vestuário infantil. Depois de uma semana cheia de entregas, a maior cooperativa da cidade, SAET (Sociedade de Agropecuária e Empreendedorismo de Tumingapurubá), pediu para que ela produzisse bodies para bebês, pois o departamento de Recursos Humanos da cooperativa queria dar um de presente para toda funcionária ou cooperada gestante que entrasse de licença maternidade. 

Para confeccionar os bodies, existem $n$ caixas com tiras de comprimento $c_i$. Sila pediu para que seu marido, Amário, cortasse essas tiras em tamanhos iguais, sem deixar sobras e procurando deixar as tiras o mais compridas possível. Para fazer isso, Amário criou o seguinte processo para juntar duas caixas adjacentes:

\begin{itemize}
    \item Pegar duas caixas adjacentes $i$ e $i+1$ com tiras de comprimentos $c_i$ e $c_{i+1}$;
    \item Cortar todas as tiras das duas caixas para que os pedaços fiquem com tamanho MDC$(c_i, c_{i+1})$, onde MDC é a operação Máximo Divisor Comum;
    \item Substituir as duas caixas $i$ e $i+1$ por uma caixa com as tiras de tamanho MDC$(c_i, c_{i+1})$ recém cortadas.
\end{itemize}

Para ter uma noção de quanto Amário irá demorar, Sila mandou que você determinasse qual é o mínimo de vezes que o processo de junção terá de ser realizado para que, ao final, todas as tiras das caixas remanescentes tenham o mesmo tamanho. Note que a quantidade de tiras não importa, mas sim o tamanho delas em cada caixa.

Por exemplo, se houverem $n=5$ caixas, cada uma com tiras de comprimento $c=[10,30,12,42,2]$, o processo pode ser realizado um mínimo de 3 vezes até que, ao final, todas as caixas tenham tiras de mesmo comprimento e o processo não possa mais ser realizado. Uma das formas de juntar as caixas 3 vezes, satisfazendo o pedido de Sila, é:

\begin{center}
    $[\textcolor{red}{10},\textcolor{red}{30},12,42,2] \rightarrow [10,\textcolor{red}{12},\textcolor{red}{42},2] \rightarrow [\textcolor{red}{10},\textcolor{red}{6},2] \rightarrow [2,2]$
\end{center}

Em vermelho, estão o comprimento das tiras das duas caixas escolhidas para realizar a junção.

\subsection*{Entrada}
 
A primeira linha da entrada contém o número de caixas $n$ ($2\leq n\leq 10^5$).
 
A próxima linha consiste em $n$ inteiros $c_i$ ($1\leq c_i \leq 10^5$), o tamanho das tiras dentro da caixa $i$.
 
\subsection*{Saída}
A saída deverá conter uma linha com o número de processos de junção realizados, atendendo à ordem de Sila.

%----- Exemplo 1 -----%
\newpage
\begin{table}[!h]
\centering
\hspace{-2cm}
\begin{tabular}{|l|l|}
\hline
\begin{minipage}[t]{5.5in}
\textbf{Exemplo de entrada}
\begin{verbatim}
5
10 30 12 42 2
\end{verbatim}
\vspace{1mm}
\end{minipage}
&
\begin{minipage}[t]{1.5in}
\textbf{Exemplo de saída}
\begin{verbatim}
3
\end{verbatim}
\vspace{1mm}
\end{minipage} \\
\hline
\end{tabular}
\end{table}

%----- Exemplo 2 -----%
\begin{table}[!h]
\centering
\hspace{-2cm}
\begin{tabular}{|l|l|}
\hline
\begin{minipage}[t]{5.5in}
\textbf{Exemplo de entrada}
\begin{verbatim}
5
16 4 4 2 8
\end{verbatim}
\vspace{1mm}
\end{minipage}
&
\begin{minipage}[t]{1.5in}
\textbf{Exemplo de saída}
\begin{verbatim}
4
\end{verbatim}
\vspace{1mm}
\end{minipage} \\
\hline
\end{tabular}
\end{table}

%----- Exemplo 3 -----%
\begin{table}[!h]
\centering
\hspace{-2cm}
\begin{tabular}{|l|l|}
\hline
\begin{minipage}[t]{5.5in}
\textbf{Exemplo de entrada}
\begin{verbatim}
3
1 1 100000
\end{verbatim}
\vspace{1mm}
\end{minipage}
&
\begin{minipage}[t]{1.5in}
\textbf{Exemplo de saída}
\begin{verbatim}
1
\end{verbatim}
\vspace{1mm}
\end{minipage} \\
\hline
\end{tabular}
\end{table}
