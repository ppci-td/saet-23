    Carlos e Pedro gostam muito de futebol e querem jogar uma partida um contra o outro. Como ambos são goleiros, eles precisam de jogadores no ataque e na defesa para completar o seu time.

    Para isso, eles tem uma lista de jogadores interessados em participar da partida, contendo os
pontos de ataque e de defesa de todos. Quem tem seus pontos de defesa maior que os de ataque, joga na defesa,
caso contrário, joga no ataque.

    Para a divisão ser justa, a escolha dos jogadores será feitas em rodadas. Carlos e Pedro escolherão um jogador por rodada.
Na primeira rodada Carlos escolhe um jogador primeiro, na segunda Pedro escolhe primeiro e vão intercalando até acabarem os jogadores. Caso o numero de jogadores
seja impar, um não sera escolhido.

    A cada rodada a escolha é feita de forma bem simples:
\begin{itemize}
    \item Quem tiver a maior quantidade de pontos somados é escolhido;
    \item Se alguém do ataque e da defesa tem a mesma quantidade de pontos, ambos tem preferência pelo atacante;
    \item Caso tenha mais jogadores do ataque escolhidos, se possivel, um jogador da defesa será escolhido;
    \item Caso tenha mais jogadores da defesa escolhidos, se possivel, um jogador do ataque será escolhido;
\end{itemize}

\subsection*{Entrada}

A primeira linha da entrada contém o número de jogadores a serem escolhidos $N$ ($2\leq N\leq 10^5$).
Cada uma das $N$ linhas seguintes contém 2 inteiros: $K_i$ e $M_i$ ($0\leq K_i\leq 99, 0\leq M_i\leq 99$), sendo, respectivamente. os pontos de ataque
e de defesa do jogador $i$.  

\subsection*{Saída}

A saída deverá conter uma linha com a média do time com a maior quantidade de pontos somados, com truncamento em duas casas decimais.

%----- Exemplo 1 -----%
\newpage
\begin{table}[!h]
\centering
\begin{tabular}{|l|l|}
\hline
\begin{minipage}[t]{3in}
\textbf{Exemplo de entrada}
\begin{verbatim}
4
10 20
30 10
10 50
10 5
\end{verbatim}
\vspace{1mm}
\end{minipage}
&
\begin{minipage}[t]{3in}
\textbf{Exemplo de saída}
\begin{verbatim}
37.50
\end{verbatim}
\vspace{1mm}
\end{minipage} \\
\hline
\end{tabular}
\end{table}

%----- Exemplo 2 -----%
\begin{table}[!h]
\centering
\begin{tabular}{|l|l|}
\hline
\begin{minipage}[t]{3in}
\textbf{Exemplo de entrada}
\begin{verbatim}
6
62 1
79 65
71 71
 8 91
71 99
20 24
\end{verbatim}
\vspace{1mm}
\end{minipage}
&
\begin{minipage}[t]{3in}
\textbf{Exemplo de saída}
\begin{verbatim}
125.00
\end{verbatim}
\vspace{1mm}
\end{minipage} \\
\hline
\end{tabular}
\end{table}

%----- Exemplo 3 -----%
\begin{table}[!h]
\centering
\begin{tabular}{|l|l|}
\hline
\begin{minipage}[t]{3in}
\textbf{Exemplo de entrada}
\begin{verbatim}
10
61 55
49 81
96 23
 2 59
56 41
33 20
93 16
72 20
65 66
58 94
\end{verbatim}
\vspace{1mm}
\end{minipage}
&
\begin{minipage}[t]{3in}
\textbf{Exemplo de saída}
\begin{verbatim}
117.40
\end{verbatim}
\vspace{1mm}
\end{minipage} \\
\hline
\end{tabular}
\end{table}