\documentclass[12pt,oneside]{article} % Uma Coluna e lingua portuguesa
%\usepackage[T1]{fontenc}        % Permite digitar os acentos de forma normal
\usepackage[utf8]{inputenc}
%\usepackage[english]{babel}
\usepackage[portuges,brazil]{babel}
%\usepackage[latin1]{inputenc}
\usepackage[dvips]{graphicx}    % Permite Gráficos
%\usepackage{times}    % Fonte Times
\usepackage{fancyhdr}
\usepackage{array}
\usepackage{multicol}
\usepackage[colorlinks=true,linkcolor=blue,urlcolor=blue]{hyperref}
\usepackage{nomencl}    % glossario
\usepackage{amssymb}
\usepackage{amsmath}
\usepackage[compact]{titlesec}
\usepackage{wrapfig}
\usepackage{color}

%=======================================================================

% Hifenização das palavras desconhecidas pelo LaTeX
%\hyphenation{}
\paperheight    297mm
\paperwidth     210mm
\voffset         -15mm
\headheight      15pt %% tamanho de letra
\headsep         5mm  %% para o início do texto
\oddsidemargin  -3.0mm
\evensidemargin -3.0mm
\textwidth      167.0mm
\topmargin      005.0mm
\textheight     240.0mm
\footskip       10.0mm

\title{SAET 2023 - Maratona de Programação}

\author{Maratona de Programação}
\date{26 de Outubro de 2022}
\usepackage{indentfirst}
\usepackage{subfig}

\parindent=0pt
\setlength{\parskip}{7pt plus 1pt minus 2pt}
\titlespacing{\section}{0pt}{*0}{*0}
\titlespacing{\subsection}{0pt}{*0}{*0}
\titlespacing{\subsubsection}{0pt}{*0}{*0}

\begin{document}

\begin{center}
\textbf{\Huge SAET 2023 - Maratona de Programação} \\
\vspace{0.2cm}
\textit{26 de Outubro de 2022} \\
\vspace{1.0cm}
%\textbf{Sevidor BOCA:} \\
%\texttt{\large http://maratona.c3sl.ufpr.br/boca/} \\
%\vspace{1.0cm}
\begin{figure}[h!]
	\centering
 \includegraphics[scale=0.95]{capa.png}
\end{figure}
\vspace{1.0cm}
%\textbf{Organizadores:}\\
%{\small Flávio Zavan} \\
%{\small Ricardo Oliveira} \\
\vspace{1.0cm}
\end{center}

\clearpage

\pagestyle{fancy}
\renewcommand{\footrulewidth}{0.7pt}
\renewcommand{\headrulewidth}{0.7pt}
\lhead{SAET 2023 - Maratona de Programação}
%\chead{Maratonas de Programação}
\rhead{26 de Outubro de 2023}
\cfoot{\thepage}

\newpage

% Espaco para o create-zips.sh nao achar
  \section*{Instruções Importantes}

\begin{itemize}

    \item Use a opção \textbf{Runs} para enviar suas soluções. Os problemas podem resolvidos em qualquer ordem e
    linguagem (dentre C, C++ e Python, independentemente do problema);

    \item Suas soluções serão testadas com várias entradas,
    além das dada como exemplo. Por isso, sua solução pode não ser
    aceita mesmo se funcionar para os exemplos dados. Certifique-se que ela
    funciona para todas as entradas possíveis;

    \item A saída gerada deve ser \textit{exatamente} conforme
    especificada. Em particular, \textbf{não} imprima instruções (``digite um
            número'', ``a resposta é'', etc);

    \item É garantido que todas as entradas usadas para teste estarão de acordo
    com o enunciado, não sendo necessário testar se são válidas;

    \item Ao enviar uma solução, o sistema irá responder uma das
    seguintes respostas:
    \begin{itemize}
        \item \verb|Not answered yet|: a solução está sendo corrigida.
        Aguarde um pouco e atualize a página;
        \item \verb|YES|: solução aceita. Parabéns!
        \item \verb|Wrong Answer|: a saída impressa pelo seu programa não é a
        saída correta esperada, para alguma entrada de teste;
        \item \verb|Presentation Error|: a saída impressa está correta, exceto
        por espaços em branco e/ou quebras-de-linha faltando/sobrando;
        \item \verb|Time Limit Exceeded|: o tempo de execução do seu programa
        ultrapassou o tempo limite estipulado para o problema (ver tabela
        abaixo). O tempo de execução da sua solução precisa ser menor;
        \item \verb|Runtime Error|: seu programa gerou algum erro em tempo de
        execução (``crashou'');
        \item \verb|Compile Error|: seu programa não compila.
    \end{itemize}

    \item Todas as linhas, tanto na entrada quanto na saída, terminam com o
    caractere de fim-de-linha ($\backslash n$), mesmo quando houver apenas uma única
    linha na entrada e/ou saída;

    \item Sua solução deve processar cada arquivo de entrada no tempo máximo
    estipulado para cada problema, dado pela seguinte tabela:

    \begin{table}[h]
    \centering
    \begin{tabular}{|c|c||c|}
    \hline
    \textbf{Problema} & \textbf{Nome} & \textbf{Tempo Limite (segundos)} \\
    \hline
    D & Drawkcabackward & 1 \\
    \hline
    \end{tabular}
    \end{table}

\end{itemize}

\newpage
\section*{D: Drawkcabackward } %tle=1
\vspace{-0.52cm}
\noindent \begin{verbatim}Arquivo: drawkcab.[c|cpp|py]\end{verbatim}
Uma das mais divertidas tarefas de um maratonista de programação é a escolha do
nome da sua equipe. Depois de muito debate, sua equipe decidiu escolher o nome
da seguinte maneira:

\begin{itemize}
    \item O nome deverá ser uma substring não vazia de uma dada string $s$;
    \item O nome deverá ser palíndrome (isto é, deve ser igual quando lido da
            esquerda para a direita e da direita para a esquerda);
    \item Cada letra do alfabeto $a$, $b$, $c$, ..., $z$ tem um \textit{valor}
    $V_a$, $V_b$, $V_c$, ..., $V_z$. O \textit{valor total} do nome escolhido é a soma dos valores de
    suas letras. O nome escolhido deverá ter o maior valor total possível.
\end{itemize}

Como exemplo, considere a string $s = $ \verb|xabaydcbbcdyz| e os valores
$V_a = 20$, $V_b = -10$, $V_c = 15$, $V_d = 11$, $V_x = 20$, $V_y = -20$ e $V_z
= 20$. Alguns nomes que poderiam ser escolhidos são:

\begin{itemize}
    \item \verb|aba|, com valor total $V_a + V_b + V_a = 20-10+20 = 30$;
    \item \verb|dcbbcd|, com valor total $11 + 15 - 10 - 10 + 15 + 11 = 32$;
    \item \verb|ydcbbcdy|, com valor total $-20 + 11 + 15 - 10 - 10 + 15 + 11
    -20 = -8$;
    \item Outras substrings palíndromes de $s$.
\end{itemize}

Neste exemplo, a substring palíndrome \verb|dcbbcd| tem o maior valor total
possível (32), e portanto este será o nome da equipe.

Dada a string $s$ e os valores de cada letra, ajude a escolher o nome da sua
equipe!

\subsection*{Entrada}

A primeira linha contém 26 valores inteiros $V_a, V_b, V_c, ..., V_z$ (entre
        $-1000$ e $1000$ cada) indicando o
valor de cada letra do alfabeto.

A segunda linha contém a string $s$ ($1 \leq |s| \leq 5 \times 10^5$), contendo
apenas letras minúsculas.

\subsection*{Saída}

Imprima uma única linha contendo o valor total do nome escolhido pela equipe.

\newpage
\begin{table}[!h]
\centering
\hspace{-2cm}
\begin{tabular}{|l|l|}
\hline
\begin{minipage}[t]{5.5in}
\textbf{Exemplo de entrada}
\begin{verbatim}
20 -10 15 11 0 0 0 0 0 0 0 0 0 0 0 0 0 0 0 0 0 0 20 0 -20 20
xabaydcbbcdyz
\end{verbatim}
\vspace{1mm}
\end{minipage}
&
\begin{minipage}[t]{1.5in}
\textbf{Exemplo de saída}
\begin{verbatim}
32
\end{verbatim}
\vspace{1mm}
\end{minipage} \\
\hline
\end{tabular}
\end{table}

\begin{table}[!h]
\centering
\hspace{-2cm}
\begin{tabular}{|l|l|}
\hline
\begin{minipage}[t]{5.5in}
\textbf{Exemplo de entrada}
\begin{verbatim}
1 -1 0 0 0 0 0 0 0 0 0 0 0 0 0 0 0 0 0 0 0 0 0 0 0 0
bbbaaaaaabbbbbbaaa
\end{verbatim}
\vspace{1mm}
\end{minipage}
&
\begin{minipage}[t]{1.5in}
\textbf{Exemplo de saída}
\begin{verbatim}
6
\end{verbatim}
\vspace{1mm}
\end{minipage} \\
\hline
\end{tabular}
\end{table}

\begin{table}[!h]
\centering
\hspace{-2cm}
\begin{tabular}{|l|l|}
\hline
\begin{minipage}[t]{5.5in}
\textbf{Exemplo de entrada}
\begin{verbatim}
7 1 1 8 1 -5 1 1 3 1 -10 1 71 -42 1 1 1 1 1 1 1 1 1 1 1 1
lazafakeekafdrawkcabackwardlaza
\end{verbatim}
\vspace{1mm}
\end{minipage}
&
\begin{minipage}[t]{1.5in}
\textbf{Exemplo de saída}
\begin{verbatim}
31
\end{verbatim}
\vspace{1mm}
\end{minipage} \\
\hline
\end{tabular}
\end{table}

\begin{table}[!h]
\centering
\hspace{-2cm}
\begin{tabular}{|l|l|}
\hline
\begin{minipage}[t]{5.5in}
\textbf{Exemplo de entrada}
\begin{verbatim}
-1 0 0 0 0 0 0 0 0 0 0 0 0 0 0 0 0 0 0 0 0 0 0 0 0 0
aaaaaaaaaaaaaaaaaaaaaaaaaaaaaaaaaaaaaaaaaaaaaaaaaaaa
\end{verbatim}
\vspace{1mm}
\end{minipage}
&
\begin{minipage}[t]{1.5in}
\textbf{Exemplo de saída}
\begin{verbatim}
-1
\end{verbatim}
\vspace{1mm}
\end{minipage} \\
\hline
\end{tabular}
\end{table}


\end{document}
